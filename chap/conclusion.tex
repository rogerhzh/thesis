% Copyright (c) 2014,2016 Casper Ti. Vector
% Public domain.

\chapter{总结} \label{chap:conclusion}

随着多核处理器技术的普及,越来越多的处理器核被集成到单个处理芯片上。虽然这大大提高了处理器并发执行任务的能力,但存储访问的瓶颈问题却日益突出。并发执行的线程由于竞争使用底层共享缓存资源,从而造成严重的性能下降。因此,如何发现、控制和缓解对高速缓存等共享资源的竞争,提高并发系统的性能与隔离性,是多核环境中一个亟需探索的重要课题。

在本文中,我们提出了一个基于部分共享的高速缓存分配优化框架(CAPS,Cache Allocation with Partial Sharing)。这是一个纯软件的多目标优化框架,相比于以往的优化方案,它的主要优势有以下几点:

\begin{itemize}
    \item 可以在较细粒度层面实现对缓存占用的控制和管理
    \item 具有目标灵活性,可以支持多种优化目标
    \item 具有良好的可扩展性,在核数较多的情况下也同样适用
    \item 可以在真实机器上运行
\end{itemize}

CAPS依赖于英特尔最新发布的高速缓存分配技术(CAT)在真机上执行分配方案。但是CAT本身是一种粗粒度的路分配技术(Way-partitioning),在核数/线程数较多时,效果会大打折扣。而CAPS通过分配之间的部分重叠,突破了CAT技术本身的这种限制,实现了细粒度的缓存分配。

CAPS的核心主要包含预测模型和分配算法两个部分。预测模型对于任意一个CAT分配方案下的多核工作负载可以预测出它们的失效率和IPC。分配算法是一个基于模拟退火的优化算法,对于一个指定的优化目标,可以产生一个优化分配方案,这个方案可能是部分重叠的,并可以直接被CAT技术所采纳。

同时CAPS可以支持多种优化目标,并且对于不同的目标,只需要改动目标估值函数,而不需要对策略算法做很大的改动。我们在真实环境中对CAPS进行了大量的实验,结果显示CAPS生成的部分重叠方案在五种优化指标上都能起到良好的效果。

目前CAPS只支持静态优化,需要离线对程序进行采样分析。在未来,我们希望把CAPS拓展到在线场景中,可以根据当前执行的程序特征,动态地调整优化方案。
